\chapter{Verfahrensbeschreibung}
\label{chap:Verfahrensbeschreibung}
\section{Vorgehensweise}
% Beschreibung wie das Programm und der implementierte Algorithmus vorgehen.

\subsection*{Sonderfälle}

Die oben genannten Sonderfälle werden wie folgt behandelt:
\begin{description}
	\item[Sonderfall Bezeichnung] Sonderfall Beschreibung und Behandlung 
\end{description}

\subsection*{Fehlerfälle}

Die oben genannten Fehlerfälle werden wie folgt behandelt:
\begin{description}
	\item [Fehlerfall Bezeichnung] Fehlerfall Beschreibung und Behandlung
\end{description}

In den Fällen, wo nach einem Fehler weitergearbeitet wird, wird auch weiter auf andere Fehler geprüft. Sollten Fehler entdeckt werden und die Verarbeitung fortgesetzt, so wird in der Ausgabe eine Hinweismeldung erscheinen, dass trotz eines Fehlers die Verarbeitung fortgeführt wurde.

\section{Gesamtsystem}
% Unterteilt in verschiedene Subsections gemäß Packages
\subsection{Main-Funktion}

\subsection{Model-Klassen}

\subsection{IO-Klassen}

\clearpage
\subsection{Controller-Klassen}

\section{Datenfluss}
%Grafische Darstellung der Kommunikation unter den Klassen und Objekten
% Sequenzdiagramm
%\begin{figure}[!h]
%	\centering
%	\includegraphics[width=1.1\textwidth]{./Graphics/Datenfluss}
%	\caption{Datenfluss unter den Klassen}
%\end{figure}
\cleardoublepage